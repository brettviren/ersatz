
% TEXINPUTS=.:$HOME/git/bvtex: latexmk -pdf <main>.tex
\documentclass[xcolor=dvipsnames]{beamer}

\input{defaults} 
\input{beamer/preamble}

\setbeamertemplate{navigation symbols}{} %
%\setbeamertemplate{background}[grid][step=1cm]

\usepackage{siunitx} \usepackage{xmpmulti}
\usepackage[export]{adjustbox}

\usepackage[outline]{contour} \usepackage{tikz}
\usetikzlibrary{shapes.geometric, arrows} \usetikzlibrary{positioning}

\definecolor{bvtitlecolor}{rgb}{0.98, 0.92, 0.84}
\definecolor{bvoutline}{rgb}{0.1, 0.1, 0.1}

\renewcommand{\bvtitleauthor}{Brett Viren}
\renewcommand{\bvtit}{Ersatz}
\renewcommand{\bvtitle}{\LARGE Ersatz\\\large A Discrete Event Simulation of\\Distributed Applications}
\renewcommand{\bvevent}{protoDUNE DAQ Sim\\ \small 8 Jul 2016} 
\renewcommand{\bvbeamerbackground}{}

\begin{document} 

\input{beamer/title.tex}
%\input{beamer/toc.tex}

\section{Caveat}

\begin{frame}
  \begin{center}
    \textbf{\Huge CAVEAT AUDITORIUM}

    \vspace{10mm}

    This is still very much a work in progress.

    \vspace{10mm}
    Anyone interested in getting involved?  

    You are very welcome!
  \end{center}

  \vfill

  $\rightarrow$ for now, still heavy on the concepts, \\light on the results....
\end{frame}

\section{The Problem}

\begin{frame}
  \tableofcontents[currentsection,hideothersubsections]
\end{frame}

\begin{frame} 
  \frametitle{protoDUNE/SP Data Scenarios \href{http://docs.dunescience.org:8080/cgi-bin/ShowDocument?docid=1086}{DocDB 1086-v6}}
  
  \begin{itemize}\footnotesize
  \item Capture quantitative assumptions and implications and guide design.
    \begin{itemize}\scriptsize
    \item (Please let's maintain \textbf{one} source for this kind of info)
    \end{itemize}
  \item \textbf{Recently revised} (upward) by Tom Junk with help from FNAL Computing.
  \item 25-50Hz, 5ms, 6APA, 2-4$\times$ compression, 25-50M events.
  \item 25-50TB buffer disk, 30-60 parallel HDD writes, 1.5-3.0 GByte/sec.
    \begin{itemize}  \scriptsize
    \item Instantaneous - but taking just as many cosmics as beam between spills!
    \end{itemize}
  \end{itemize}

  \vfill

  \footnotesize

  To handle this data, I wonder: \textbf{what computing and networking
  elements are needed, of what type, how many and how are they
  interconnected?} 

  \vfill

  There is a relatively high level of complexity, the assumptions have
  spanned at least an order of magnitude, and keep changing
  $\Rightarrow$ need an efficient and quantitative way to explore the
  configuration space.

  \vfill
  $\rightarrow$ Simulation!
\end{frame}

\section{The Conceptual Model}

\begin{frame}
  \tableofcontents[currentsection,hideothersubsections]
\end{frame}

\begin{frame}[fragile]
  \frametitle{Conceptual System Model}


  \begin{columns}
    \begin{column}{0.4\textwidth}
      \includegraphics[width=\textwidth]{generic.pdf}      

      \scriptsize 

      Logically, the graph is made of fully-connected layers.

      Physically, there are switch, NIC and computer constraints.

    \end{column}
    \begin{column}{0.6\textwidth}
      \footnotesize
      Model joint online/offline context as \textit{directed acyclic graph} of
      \textit{functional nodes} consuming, processing and producing \textit{discrete units of data}.
      \normalsize
      \vspace{2mm}

      The scope of the model may include:
      \begin{itemize}\footnotesize
      \item Digital readout electronics,
      \item DAQ elements (ie, artDAQ nodes),
      \item Buffer storage units,
      \item Prompt processing jobs for QA/QC/commissioning.
      \item Networking, SATA bus.
      \item[$\rightarrow$] \textbf{don't care about the actual content} of the data
        or its processing, just \textbf{data sizes, rates, processing time}, etc.
      \end{itemize}
      
    \end{column}
  \end{columns}


\end{frame}

\begin{frame}
  \frametitle{Data model} 

  \footnotesize
  Datum\footnote{gcide v.0.48} \textit{a single piece of information; especially a
    piece of information obtained by observation or experiment; --
    used mostly in the plural.} 
  \normalsize

  Except here, it really is singular!

  \begin{itemize}
  \item A \textbf{unit of data is discrete}, no open ended streams.
  \item Examples:
    \begin{itemize}
    \item A fragment of a readout from a Board Reader
    \item A single readout from an Event Builder
    \end{itemize}
  \item A datum has effectively \textbf{just two numbers}:
    \begin{enumerate}
    \item a size in Bytes
    \item an identifier.
    \end{enumerate}
  \item Lifetime: produced by one node, consumed by another.
  \end{itemize}
  \vspace{-10mm}
  \begin{center}
    \includegraphics[height=2cm]{prodcons.pdf}
  \end{center}
  \vspace{-10mm}
\end{frame}

\begin{frame}
  \frametitle{Node model}


  \begin{center}
  \vspace{-10mm}
  \includegraphics[width=0.9\textwidth]{onenode.pdf}    
  \vspace{-15mm}
  \end{center}

  Each node modeled as having:
  \begin{itemize}\footnotesize
  \item An \textbf{address}
  \item \textbf{Input bandwidth} limit and datum \textbf{buffer depth}.
  \item Per-datum \textbf{processing latency}.
  \item Data \textbf{reduction/inflation} factor.
  \item \textbf{Output bandwidth} limit and datum \textbf{buffer depth}.
  \item A \textbf{routing strategy} for addressing output datum to a downstream node.
  \end{itemize}

  \footnotesize
  Notes:
  \begin{itemize}
  \item Switch bandwidth still applies, but node doesn't ``care''.
  \item One node is not (necessarily) equated with one computer host.
  \end{itemize}

\end{frame}

\begin{frame}
  \frametitle{Host model}
  Model a computer box to constrain nodes.
  \begin{itemize}
  \item Assert NIC RX/TX maximum bandwidth constraints.
  \item Restrict number of node I/O buffers (aka RAM constraint).
  \item Limit minimum processing time (aka CPU constraint).
  \end{itemize}

  \vfill
  \footnotesize

  I've not thought enough about this particular element yet....

\end{frame}

\begin{frame}
  \frametitle{Switch model}
  Model a network switch:

  \begin{itemize}
  \item Bandwidth limited at both RX and TX ports  (eg, @ 1Gbps).
  \item Full-duplex ports, assume infinite switch fabric bandwidth.
  \item Shared bandwidth.
    \begin{itemize}
    \item Invented a simple, iterative load balancing algorithm.
    \item Competing transfers between shared TX and RX ports.
    \item Each stream goes as fast as possible subject to fair-share.
    \end{itemize}
  \item Preemptive.
    \begin{enumerate}
    \item A new transfer interrupts the switch.
    \item All in-progress transfers get their progress updated.
    \item New stream added and overall bandwidth load-balanced.
    \item Continue until interrupted or next pending transfer completes.
    \end{enumerate}
  \end{itemize}

  (This has been implemented and tested.)
\end{frame}

\begin{frame}
  \frametitle{Discrete Event Simulation}

  Basic idea:
  \begin{itemize}
  \item Total system state changes in \textbf{discrete steps}.
  \item Changes occur based on an ``\textbf{event}''. 
    \begin{itemize}
    \item Most events defined in terms of a ``\textbf{timeout}'' primitive.
    \item[$\rightarrow$] eg: ``\textit{do this thing in 3 seconds from now}''
    \end{itemize}
  \item Change state by executing associated ``\textbf{event callbacks}''.
  \end{itemize}

  \vspace{5mm}

  Example: transfer a datum on a network link:
  \begin{enumerate}\footnotesize
  \item Get available \textbf{bandwidth} and fixed \textbf{latency} for the link.
  \item Get the \textbf{size} of the datum.
  \item Set \textbf{timeout}(\texttt{now + latency + size/bandwidth}).
  \item Raise event ``transfer complete'' and trigger associated callbacks.
  \end{enumerate}

  \vspace{5mm}
  $\rightarrow$ tl;dr: focus on connecting detailed, local event callbacks
  and let the system work out the overall complex behavior.

\end{frame}

\section{The Software}

\begin{frame}
  \tableofcontents[currentsection,hideothersubsections]
\end{frame}

\begin{frame}
  \frametitle{Ersatz}

  \begin{center}
    \url{https://github.com/brettviren/ersatz}
  \end{center}

  \begin{itemize}
  \item Based on \href{https://simpy.readthedocs.io/}{SimPy 3} and Python 3.
    \begin{itemize}
    \item Asynchronous co-routines but single-threaded.
    \item Python generators and heavy use of \texttt{yield}.
    \end{itemize}
  \item \textbf{Ersatz} package now provides, (or will, \textit{Real Soon Now}):
    \begin{itemize}
    \item A shared bandwidth, preemptive network switch (done). 
    \item Generic, parametrized node (started).
    \item Graph description and layout (todo).
    \item Configuration and command line interface (started).
    \item State monitoring and visualization services (rudimentary).
    \item Many unit tests and examples (already)
    \item Documentation (there, but lagging)
    \end{itemize}
  \end{itemize}
  Developers welcome.  Users need to wait a bit.

\end{frame}

\begin{frame}
  \begin{center}
    Watch this space.
  \end{center}
\end{frame}

\end{document}


%%% Local Variables:
%%% mode: latex
%%% TeX-master: t
%%% End:
